\documentclass[12pt,a4paper]{article}
\usepackage[utf8]{inputenc}
\usepackage[portuguese]{babel}
\usepackage[T1]{fontenc}
\usepackage{amsmath}
\usepackage{amsfonts}
\usepackage{amssymb}
\usepackage{makeidx}
\usepackage{graphicx}
\usepackage[left=3cm,right=2cm,top=3cm,bottom=2cm]{geometry}

\usepackage{float}
\usepackage{indentfirst}
\usepackage{setspace}
\usepackage{breakurl}
\usepackage{hyperref}
\usepackage{xcolor}
\usepackage{ulem}
\usepackage{seqsplit}


\newcommand{\link}[1]{{\color{blue}\url{#1}}}
\newcommand{\blue}[1]{\textcolor{blue}{#1}}

\begin{document}
	\singlespacing
	\begin{titlepage}
		\begin{center}
			\begin{figure}[!htb]
				\center
				\includegraphics[scale=0.25]{pictures/Sigla.pdf} 
			\end{figure}
			{\bf UNIVERSIDADE FEDERAL DE SANTA CATARINA}\\[0.2cm]
			{\bf CENTRO TECNOLÓGICO}\\[0.2cm]
			{\bf DEPARTAMENTO DE INFORMÁTICA E ESTATÍSTICA}\\[5.5cm]
			{\bf \large EXERCÍCIOS DE CRIPTOGRAFIA SIMÉTRICA, HASH, MAC, PBKDF E CRIPTOGRAFIA AUTENTICADA EM JAVA}\\[3.6 cm]
			{Ivo Guilherme Kurtz Bohm}\\	
			{Sadi Júnior Domingos Jacinto}\\[1cm]
			{Professora orientadora: Carla Merkle Westphall}\\[4.1 cm]
			{Florianópolis}\\[0.2cm]
			{2020}
		\end{center}
	\end{titlepage}
	
	\section{\normalsize QUESTÕES}
		\begin{enumerate}
			\item[1.] Abra o \textbf{projeto2CodigoLivro} e teste o seu funcionamento. Responda:
			
			\begin{enumerate}
				\item[1.1.] Qual algoritmo é usado no código? Em qual modo?\\
				Resposta: O algoritmo usado é o \textit{AES} no modo \textit{CBC}.
				
				\item[1.2.] Explique o que faz o método \textit{generateKey} da classe \link{https://docs.oracle.com/javase/7/docs/api/javax/crypto/KeyGenerator.html.KeyGenerator}\\
				Resposta: Gera uma chave simétrica, podendo a chave ser gerada independente de um algoritmo ou de maneira específica de um algoritmo.
				
				\item[1.3.] Explique como são usados os métodos \textit{init}, \textit{update} e \textit{doFinal} para cifrar e para decifrar os dados nesse código. Leia a documentação e entenda bem o funcionamento desses métodos.\\
				Resposta:\\
				\begin{itemize}
					\item \textbf{\textit{init}:} Inicializa a cifra com uma chave e um conjunto de parâmetros de algoritmo, podendo ser inicializada para uma das quatro seguintes operações: criptografia, decodificação, embalagem da chave ou desembrulhamento da chave, dependendo do valor do parâmetro \textit{opmode}.\\
					No exemplo, a cifra foi inicializada usando \textbf{DECRYPT\_MODE} (decifrar), com uma chave gerada previamente e um IV aleatório. 
					
					\item \textbf{\textit{update}:} Usado para continuar uma operação de criptografia ou decriptação de múltiplas partes (dependendo de como a cifra foi inicializada), processando outra parte de dados. Retorna o número de \textit{bytes} armazenados na saída.
					
					\item \textbf{\textit{doFinal}:} Finaliza a operação de criptografia ou decriptografia de múltiplas partes,  dependendo de como a cifra foi inicializada. Os dados de entrada que podem ter sido armazenados em \textit{buffer} durante uma operação de atualização anterior são processados. Ao terminar, este método reinicia a cifra para o estado em que se encontrava inicialmente através de uma chamada para o init. Ou seja, o objeto é reinicializado e está disponível para criptografar ou decodificar (dependendo do modo de operação que foi especificado na chamada ao \textit{init}) mais dados.
				\end{itemize}
			\end{enumerate}
			
			\item[2.] Crie um programa que permite ao usuário entrar com uma string pelo teclado, o programa cifra a string e mostra a string cifrada na tela. O código deve ``sortear'' uma boa chave e IV. Use o modo CTR (counter) do algoritmo AES para cifrar. Use o projeto3Aes para auxiliar.
			
			\item[3.] Nesse projeto você irá programar dois sistemas de decifragem, um usando o AES em \textbf{modo CBC} e outro usando o AES no \textbf{modo contador} (\textit{counter mode} -- CTR). Em ambos os casos um IV de 16 bytes é escolhido de forma aleatória. Para o modo CBC use o esquema de padding PKCS5. Para o modo CTR use NoPadding.
			
				Inicialmente iremos testar apenas a função de decifragem. Use o projeto3Aes para auxiliar a responder as questões. Nas questões seguintes você recebe uma chave AES, um IV e um texto cifrado (ambos
codificados em hexa) e o seu objetivo é recuperar o texto plano/texto decifrado. \uline{A resposta de cada questão é o texto decifrado (frase legível).}
				
				\begin{enumerate}
					\item[3.1.] 
						\begin{itemize}
							\item Chave CBC: 53efb4b1157fccdb9902676329debc52
							\item IV: d161fbaa4c64ecf7d2c4abd885751273
							\item Texto cifrado em modo CBC: \seqsplit{701f7fa45d9bb922c3cb15a519ba40ede1769eb753650886d6e69ebcad9c2816002679896a65a921d25
e00793078474e3dbeca9a2838031c490e5ae9d1ea143f}
						\end{itemize}
						
						\textbf{Resposta:} Modo CBC (Cipher Block Chaining) do AES - cifragem encadeada.
						
					\item[3.2.]
						\begin{itemize}
							\item Chave CTR: a05e2679204241af07f6857d150a1fcd
							\item IV: 468ce1126a37b07138e78eab48344712
							\item Texto cifrado em modo CTR: \seqsplit{36466b5fddcfcb1b8a9479eb8c489e7139a3c35020b1e5ee808b39ff18b6abd812afe7dbbca40e15df391
a7c07ece1c8e10a49368b86a946c8379cd8fa01a47f1956671144b0ca18a4c812cde8f7b9}
						\end{itemize}
						
						\textbf{Resposta:} Modo CTR (counter) - cifra a contagem do IV e faz XOR com bloco de texto plano.
				\end{enumerate}
			
			\item[4.] Crie um programa que recebe duas strings pelo teclado, calcula o hash (resumo criptográfico) e o MAC de cada uma das strings escrevendo o resultado na tela. Teste e explique o funcionamento do programa com entrada de strings iguais e depois com entrada de strings diferentes.
			
			\item[5.] Crie um programa que recebe uma string pelo teclado e cifra a string usando CRIPTOGRAFIA AUTENTICADA (AES no modo GCM). O programa também deve gerar uma boa chave usando PBKDF2.
		\end{enumerate}	
\end{document}